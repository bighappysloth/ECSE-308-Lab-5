\documentclass[../../main.tex]{subfiles}

\begin{document}
\problem{51: What is the content length of the second response? Explain. \lstinline{http://www.tsp.ece.mcgill.ca/}}
\begin{wts}
What is the content length of the second response? Explain. \lstinline{http://www.tsp.ece.mcgill.ca/}
\end{wts}
\begin{proof}
Consider the following graphic.\\
\includegraphics[width=\textwidth]{subfiles/images/ECSE_308_Lab_5_1_SUPA_PAGE2_4_Image33.png}
\begin{wts}
    Let $\binom{n}{k}$ denote the number of $k$-element subsets of an $n$-element set. Prove that if $n<k$, then $\binom{n}{k}=0$, and that
    \[
    \binom{n+1}{k+1} = \binom{n}{k+1} + \binom{n}{k}
    \]
\end{wts}
\begin{proof}
    We begin with some abstract notation.
    \begin{itemize}
        \item $J_n = \{1,2\cdots,n\}\subseteq\nat^+$,
        \item $|E|\defined\sum_{x\in E}1$, the counting measure on $E$.
        \item $X$ is any set where $|X|\geq 2$,
        \item $A\subseteq X$, $|A||A^c|\neq 0$, this implicitly means that neither set is empty,
        \item $\Omega_n= \{f:J_n\to X\}$,
        \item For every $f\in\Omega_n$, $f_{J_{n-1}}$ denotes the restriction of $f$ onto $J_{n-1}$
    \end{itemize}
    With these definitions, it is clear that $\binom{n}{k}$, for every $n,k\in\nat$, 
    \[
    \binom{n}{k}=\biggl|\biggl\{f\in\Omega_n,\: |f^{-1}(E)|=k\biggr\}\biggr|
    \]
    Clearly, if $f\in\Omega_{n+1}$ and $|f^{-1}(E)|=k+1$, \begin{itemize}
        \item If $f^{-1}(E)\cap J_{n+1}\setminus J_{n}=\varnothing$, then $| f_{J_n}^{-1}(E)|=|f^{-1}(E)| =k+1$,
        \item If $f^{-1}(E)\cap J_{n+1}\setminus J_n\neq\varnothing$, then $|f_{J_n}^{-1}(E)|=k$,
    \end{itemize}
    Then we can write $E_1 = \biggl\{f\in\Omega_{n+1},\: |f^{-1}(E)|=k+1\biggr\}$ as the disjoint union of $E_2 = \biggl\{
    f\in\Omega_{n+1},\:
    f^{-1}(E)\cap J_{n+1}\setminus J_{n}=\varnothing,\:
    f_{J_n}^{-1}(E)|=k+1\biggr\}$ and 
    $E_3 = \biggl\{
    f\in\Omega_{n+1},\:f^{-1}(E)\cap J_{n+1}\setminus J_{n}\neq\varnothing,
    \:|f_{J_n}^{-1}(E)|=k\biggr\}$.\\
    
    Also note that $E_2\equiv \{f\in\Omega_{n},\: |f^{-1}(E)| = k+1\}$ and $E_3\equiv\{f\in\Omega_{n}.\: |f^{-1}(E)| = k\}$. Since every $f\in E_2$ induces some $g\in\Omega_n$ with $|g^{-1}(E)|=k+1$, and respectively for $f\in E_3$. And for every $g\in\Omega_n$, $|g^{-1}(E)|=k+1$ , there is a corresponding $f\in E_2$ with $f_{J_n} = g$.\\
    
    Therefore $|E_2| = \binom{n}{k+1}$ and $|E_3| = \binom{n}{k}$. Since $|\cdot|$ is just the counting measure on finite sets, and $E_1$ is the disjoint union, it follows that
    \[
    |E_1| = \binom{n+1}{k+1} = |E_2| + |E_3| = \binom{n}{k+1} + \binom{n}{k}
    \]
\end{proof}
\end{proof}

\end{document}