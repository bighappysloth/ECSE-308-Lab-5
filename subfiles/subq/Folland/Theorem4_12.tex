\documentclass[../../main.tex]{subfiles}

\begin{document}
\subsubsection*{4.12}
\begin{wts}
If $X$ is a topological space, and $A$ is any non-empty set, $\{f_n\}\subseteq X^A$ is a sequence, then $f_n\to f$ with respect to the product topology if and only if $f_n\to f$ pointwise.
\end{wts}
\begin{proof}
    Suppose that $f_n\to f$ pointwise. Since the product topology $\Tau_X$ is generated from sets of the form
    \[
    \pnv{\alpha}{E_\alpha},\:E_\alpha\in \Tau_\alpha
    \]
    And by Theorem 4.4, $\Tau_X$ consists of $\varnothing, X$ and unions of finite intersections of $\pnv{\alpha}{E_\alpha}$. We claim that for every $f\in X^A$, the following is a valid neighbourhood base for $f$
    \[
    \left\{\:\bigcap_{j\leq n}\pi^{-1}_{\alpha_j}(E_{\alpha_j}),\:E_{\alpha_j}\in \Tau_{\alpha_j}\cap\nb{\pi_{\alpha_j}(f)}\right\}
    \]
    A couple things to note
    \begin{itemize}
        \item Each $E_{\alpha_j}$ is open in $X_{\alpha_j}$, so that its inverse image is also open (in $X$). Since any neighbourhood base has to be a subset of $\Tau_X$.
        \item Only finitely many intersections are involved, so each element in the above set is open in $X$.
        \item Each $E_{\alpha_j}$ is a neighbourhood of $\pi_{\alpha_j}(f)$, meaning $f\in E_{\alpha_j}^o = E_{\alpha_j}$.
        \item Last and perhaps most importantly for intuition, fix any non-empty open set $U\in \Tau_X$ then by Theorem 4.4 (or my reading of it), $U$ can be written as the union of sets like
        \[
        \bigcap_{j\leq m}\pnv{\alpha_j}{E_{\alpha_j}},\quad E_{\alpha_j}\in\Tau_{\alpha_j}
        \]
        Then applying Theorem 4.2, the family of finite intersections of $\pnv{\alpha}{E_\alpha}$ is a base for $\Tau_X$. Then, 
        \[
        N_{base}(f) = \left\{V=\bigcap_{j\leq m}\pnv{\alpha_j}{E_{\alpha_j}},\quad E_{\alpha_j}\in\Tau_{\alpha_j}, \quad f\in V\right\}
        \]
        Has to be a neighbourhood base for any $f\in X$.
    \end{itemize}
    Now to show that $f_n\to f$ in the product topology, fix any neighbourhood $U\in\nb{f}$, then $f\in U^o$, and by definition of a neighbourhood base, there exists some $E\in N_{base}(f)$ such that $f\in E\subseteq U^o$, but this $E$ is just the finite intersection of $\pnv{\alpha_j}{E_{\alpha_j}}$, then at every $\alpha_j$
    \begin{itemize}
        \item Let $N_j$ be an integer such that for every $n\geq N_j$, $\pi_{\alpha_j}(f_n)\in E_{\alpha_j}$
        \item Set $N = \sum_{j\leq m}N_j\geq N_j$ for every $j\leq m$.
    \end{itemize}
    Then for every $n\geq N$, $f_n\in E\subseteq U^o\subseteq U$ for any arbitrary neighbourhood $U$ of $f$. So $f_n\to f$ in the product topology.\\
    
    Conversely, suppose that $f_n\to f$ in the product topology, then fix any $\alpha\in A$, and for every neighbourhood $E_\alpha$ of $\pi_\alpha(f)$, $\pnv{\alpha}{E_\alpha}$ is a neighbourhood of $f$. Hence for every $\alpha\in A$, and for every neighbourhood $E_\alpha$ of $\pi_\alpha(f)$, $pi_\alpha(f_n)$ is eventually in $E_\alpha$. This completes the proof.
\end{proof}

\end{document}