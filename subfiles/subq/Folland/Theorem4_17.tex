\documentclass[../../main.tex]{subfiles}

\begin{document}
\subsubsection*{4.17}
\begin{wts}
If $X$ is a normal space, and $A$ is a closed subspace of $X$, and $f\in C(A)$, then there exists an $F\in C(X)$ such that $F$ extends $f$.
\end{wts}
\begin{proof}
First we suppose that $f$ is real valued, so $f\in C(X, \mathbb{R})$. And define a $g\in C(A, (-1, +1))\subseteq C(A, [-1,+1])$, using
\[
g= \dfrac{f}{1+|f|}
\]
Since $g$ satisfies the assumption of Theorem 4.16 (note that we do not require $g$ to be injective), there exists a $G\in C(X, [-1, +1])$ such that $G|_A = g$. Since the set $\{-1,+1\}$ is closed in $\mathbb{R}$, $G^{-1}(\{-1,+1\})$ is closed as well. Since $G^{-1}((-1,+1))\subseteq A$, this makes $A$ and $B=^{-1}(\{-1,+1\})$ disjoint closed sets in $X$.\\

By Urysohn's Lemma, there exists a continuous function $h\in C(X, [0,1])$ such that $h|_B = 0$ and $h|_A =1$, so that the product $|hG|< 1$ for all $x\in X$. We can think of this $h$ as a continuous indicator function that filters out the parts we do not want, namely $G^{-1}\{-1,+1\}$. Now define $F$ in the following manner, since division is permissible
\[
F = \dfrac{hG}{1-|hG|}
\]
We will show that $F|_A = g/(1-|g|) = f$ indeed. Since $|g| = \frac{|f|}{1+|f|}$, and $g(1+|f|) = f$ implies that $g/(1-|g|) = f$, because $g\in C(A,(-1,+1))$ This completes the proof for any $f\in\mathbb{R}$ if $f\in C(A)$, then
\begin{enumerate}
    \item $\Re(f) = f_1\in C(A,\mathbb{R})$
    \item $\Im(f) = f_2\in C(A,\mathbb{R})$
\end{enumerate}
And by our previous argumentation, there exists two functions in $C(X, \mathbb{R})$ that extends $f_1$ and $f_2$, and $F_1 + iF_2 = f$ on $A$ and $F_1 + iF_2\in C(X)$, and the proof is complete.
\end{proof}
\end{document}