\documentclass[../main.tex]{subfiles}
\begin{document}
\providecommand{\xn}{\{x_n\}}
\providecommand{\least}{\operatorname{least}}
\providecommand{\lip}{\operatorname{Lip}}
\subsubsection*{3}
\begin{wts}
    Let $f:A\to B$, and $g:B\to C$ be $\lip$ functions from and to metric spaces. Prove that $g\circ f$ is $\lip$, with constant $L_g\cdot L_f$ where $L_g$, and $L_f$ are constants for $g$ and $f$ respectively.
\end{wts}
\begin{proof}
    Suppose both $L_g$ and $L_f$ are strictly positive, then for every $\varepsilon_C>0$, 
    \[E\subseteq B,\,\diam(E)\leq \dfrac{\varepsilon_C}{L_g}\implies \diam g(E)\leq \varepsilon_C\]
    and since $f$ is Lipschitz, it is clear that
    \[F\subseteq A,\,\diam(F)\leq\dfrac{\varepsilon_C}{L_g\cdot L_f}\implies \diam f(F)\leq \dfrac{\varepsilon_C}{L_g}\]
    and
    \[\diam (g\circ f)(F)\leq \varepsilon\]
    from this, we conclude $g\circ f$ has constant $L_g\cdot L_f$. Now suppose either $L_g$ or $L_f$ is $0$, then take $\sqrt{\varepsilon}>0$, and repeat the same proof as above, then
    \begin{align*}\diam (g\circ f) (F)&\leq (L_g + \sqrt{\varepsilon})(L_f + \sqrt{\varepsilon})\diam(F)\\
    &=(L_g\cdot L_f + \varepsilon + (L_f+L_g)\sqrt{\varepsilon})\diam(F)\end{align*}
    and sending $\varepsilon\to 0$ finishes the proof.
    \begin{remark}
        We can probably clean up the $\varepsilon$ mess by taking \[\sqrt{\varepsilon}\leq (L_f + L_g)\implies \sqrt{\varepsilon} + (L_f + L_g)\leq 2(L_f + L_g)\]
        if either $L_f$ or $L_g$ is non-zero, and take $\delta = 2\sqrt{\varepsilon}(L_f+L_g)$, and sending $\sqrt{\varepsilon}\to 0$. If both of them are $0$, then no modification is needed as the third term vanishes.
    \end{remark}
\end{proof}
\end{document}