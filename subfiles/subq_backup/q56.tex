\documentclass[../../main.tex]{subfiles}

\begin{document}
\problem{56}
\begin{wts}
What is the requested URL in the frame No. 101 ?What HTTP field contains the username and password information? What are the submitted values for the username and the password?
\end{wts}
\begin{proof}
\begin{wts}\label{theorem:piecewise bijections with disjoint ranges, inverse is the sum of piecewise inverses}
Let $f\in B^A$ satisfy the hypothesis of the previous Theorem \ref{theorem:piecewise bijective with disjoint ranges, implies bijective}, so that $(f|_{A})^{-1}$ and $(f|_{A^c})^{-1}$ both exist, and $f|_A(A)\cap f|_{A^c}(A^c)=\varnothing$, then $f^{-1} = (f|_{A})^{-1} + (f|_{A^c})^{-1} = (f^{-1})|_{B_1} + (f^{-1})|_{B_2}$, where $f|_{A}(A) = B_1$, and $f|_{A^c}(A^c)=B_2$.
\end{wts}
\begin{proof}
    Since $B_1$ and $B_2$ are disjoint, then fix any $y\in Y$. Without loss of generality, let us assume that $y\in B_1$. Then, $f^{-1}(y) = (f^{-1})|_{B_1}(y) = (f|_{A})^{-1}(y)$. This inverse is indeed well defined, since $f|_{A}$ is a bijection onto its range, then there exists a unique $x\in A$ such that applying $f$ on both sides yield
    \[
    f\left((f|_{A})^{-1}(y)\right) =f\circ (f|_A)^{-1} (y)= y
    \]
    In the same manner, fix an $x\in A$ such that $f(x)=f|_{A}(x)\in B_1$, then applying $(f|_{A})^{-1}$ on both sides
    \[
    (f|_A)^{-1}\left(f(x)\right) = (f|_{A})^{-1}\circ f (x)= x
    \]
    Therefore the inverse of $f$ can be written piecewise on two disjoint domains as follows.
    \[
    f^{-1} = f^{-1}|_{B_1} + f^{-1}_{B_2}
    \]
\end{proof}
\remark We adopt a slight abuse of notation with the 'restrictions' onto $f$, but they should be interpreted as piecewise functions. $f|_A + f|_{A^c}$ is equal to $f|_{A}\chi_A + f|_{A^c}\chi_{A^c}$ where $\chi$ is the indicator function.\\
\end{proof}

\end{document}