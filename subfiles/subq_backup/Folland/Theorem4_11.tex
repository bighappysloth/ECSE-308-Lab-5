\documentclass[../../main.tex]{subfiles}

\begin{document}
\subsubsection*{4.11}
\begin{wts}
    If $X_\alpha$ and $Y$ are topological spaces, and $X = \prod_{\alpha\in A}X_\alpha$, and $f:Y\to X$ is a mapping. Then $f$ is continuous if and only if $\pi_\alpha\circ f$ is continuous for each $\alpha\in A$.
\end{wts}
\begin{proof}
    If $\pi_\alpha\circ f$ is continuous at each $\alpha$, this means that
    \[
    \forall \alpha\in A, \:\forall E_\alpha\in\Tau_\alpha,\: f^{-1}(\pnv{\alpha}{E_\alpha})\in \Tau_Y
    \]
    But it is exactly sets of the form $\pnv{\alpha}{E_\alpha}$ which generate the weak topology for $\Tau_X$. Therefore $f$ is continuous.\\
    
    Now, suppose that $f$ is continuous, by definition of the weak topology (as it is generated by the set of inverse projections), for every $\alpha\in A$, $\pnv{\alpha}{E_\alpha}\in \Tau_X$ and by continuity of $f$, its inverse image is open in $Y$ as well.
\end{proof}
\remark The take-away intuition here is that if the range space is generated by some $\Epsilon$, then a function is continuous if and only if all inverse images of sets in $\Epsilon$ are open in the domain space. Furthermore, if the range space is endowed with the product topology (which is generated by sets of the form $\pnv{\alpha}{E_\alpha}$, where $E_\alpha\in \Tau_\alpha$), then it suffices to check all inverse images of those. And this is equivalent to checking that $\pmap{\alpha}{\cdot}\circ f$ is continuous at each $\alpha$.

\end{document}