\documentclass[../main.tex]{subfiles}
\begin{document}
\subsubsection*{7}
\begin{wts}
    Find the set of accumulation points of the following subsets of $\real$
    \begin{enumalpha}
        \item $\acc{\mathbb{Z}}=\varnothing$,
        \item $\acc{\{n^{-1},\,n\in\nat^+\}}=\{0\}$,
        \item $\acc\{(0,1)\cup\{2\}\cup[3,4]\}=[0,1]\cup[3,4]$
    \end{enumalpha}
\end{wts}

\begin{proof}[Proof of Part A]
    Let $\acc{\mathbb{Z}}$ be the set of accumulation points of $\mathbb{Z}$, we claim that $\acc{\mathbb{Z}}=\varnothing$. Indeed, suppose by contradiction that $y\in\acc{\mathbb{Z}}$, and take $\varepsilon=3^{-1}$ and it is clear that\[V_{3^{-1}}(y)\setminus\{y\}\cap\mathbb{Z}=\varnothing\] as $V_{3^{1-}}(y)$ can at most contain one element of $\mathbb{Z}$, namely $y$.
\end{proof}

\begin{proof}[Proof of Part B]
    Denote the set in question by $A$. For every neighbourhood $\varepsilon>0$ about $0$, the Archmedean Property tells us there exists some $n\geq \varepsilon^{-1}$ so
    \[0-n^{-1}\leq n^{-1}-0\leq \varepsilon\implies V_{\varepsilon}(0)\cap A\setminus\{0\}\neq\varnothing\]
    (we trade readability for conciseness), and $\{0\}\subseteq \acc{A}$. Now fix $x\neq 0$ and suppose by contradiction $x\in\acc{A}$. We have two cases to consider,
    \begin{itemize}
        \item If $x<0$, then take $\varepsilon = |x|2^{-1}$, since $0\leq \inf{A}$,
        \[
        V_\varepsilon(x)\subseteq(-\infty,0)\implies V_\varepsilon(x)\setminus\{x\}\cap A = \varnothing
        \]
        \item If $x>0$, use the Archmedean Property, and the Well Ordering Property, and denote
        \[\mathcal{N}(x)=\biggl\{n\in\nat^+,\,n>x^{-1}\biggr\}\neq\varnothing\]
        Fix $M = \least\mathcal{N}(x)\geq 1$. And write
        \[\varepsilon\defined\begin{cases}\min\biggl\{d(x,m),d(x,m-1)\biggr\}2^{-1}&m\geq 2\\
        d(x,m)2^{-1}&m=1
        \end{cases}\]
        and it follows that
        \[V_\varepsilon(x)\setminus\{x\}\cap A=\varnothing\implies x\notin \acc{A}\]
    \end{itemize}
    This proves $\acc{A}=\{0\}$
\end{proof}
\begin{proof}[Proof of Part C]
    We make the following claims
    \begin{itemize}
        \item $A\subseteq B\implies \acc{A}\subseteq \acc{B}$, let $x\in\acc{A}$, then for every $\varepsilon$-ball about $x$, its intersection with $A$ contains some $y\neq x$, $y\in A\cap V_\varepsilon(x)$, so
        \[y\in \biggl(V_\varepsilon(x)\setminus\{x\}\cap A\biggr)\subseteq V_\varepsilon(x)\setminus\{x\}\cap B\]
        \item $\acc{(A)}\cup\acc{(B)}\subseteq \acc{(A\cup B)}$, apply the first bullet point twice to the two left members, and
        \begin{itemize}
            \item $\acc{A}\subseteq\acc{(A\cup B)}$
            \item $\acc{B}\subseteq\acc{(A\cup B)}$
        \end{itemize}
        therefore $\acc{(A)}\cup\acc{(B)}\subseteq\acc{(A\cup B)}$.
    \end{itemize}
    Furthermore,
    \begin{enumerate}
        \item $\acc{(a,b)}=[a,b]$. If $x\in[a,b]$ and $x=a$ or $x=b$, so $x$ is either the supremum or infimum of $(a,b)$, and for every $\varepsilon>0$ there is an element $y\in(a,b)$ such that
        \[y\in V_\varepsilon(x)\setminus\{x\}\cap(a,b)\implies \{a,b\}\in \acc{A}\]
        Now suppose $x\in (a,b)$, and fix an arbitrary open ball about $x$ with radius $\delta>0$, since $(a,b)\cap V_\delta(x)$ is an open set that contains $x$, there exists some \[V_\varepsilon(x)\subseteq (a,b)\cap V_\delta(x)\]
        Now choose $y=x-\varepsilon2^{-1}$, and
        \[y\in V_\varepsilon(x)\implies \biggl(V_\delta(x)\setminus\{x\}\cap (a,b)\biggr)\neq\varnothing\]
        To show $\acc{(a,b)}^c\supseteq[a,b]^c$, fix any element $y\in[a,b]^c$, and this induces some ball of positive radius $\varepsilon>0$ such that
        \begin{align*}
        V_\varepsilon(y)\subseteq[a,b]^c&\implies V_\varepsilon(y)\cap[a,b]=\varnothing\\
        &\implies V_\varepsilon(y)\setminus\{y\}\cap(a,b)\subseteq\varnothing
        \end{align*}
        since $[a,b]^c$ is open in $\real$. It follows immediately that $y\notin \acc{(a,b)}$. This proves $\acc{(a,b)}=[a,b]$.
        \item $\acc{[a,b]}=[a,b]$, using the previous bullet point, and
        \[(a,b)\subseteq[a,b]\implies\acc{(a,b)}\subseteq\acc{[a,b]}\]
        So $[a,b]\subseteq\acc{[a,b]}$. To show the reverse, again fix $x\in [a,b]^c$, and by the openness of $[a,b]$, there exists \[V_\varepsilon(x)\subseteq [a,b]^c\implies V_\varepsilon(x)\setminus\{x\}\cap[a,b]=\varnothing\]
        and $x\notin \acc{[a,b]}$.
    \end{enumerate}
    Now, using the above Lemmas, and substituting values in the question, and define
    \[A = (0,1)\cup\{2\}\cup[3,4]\implies[0,1]\cup[3,4]\acc{A}\]
    $2$ is not an accumulation point of $A$, since 
    \[V_{3^{-1}}(2)\setminus\{2\}=(2-3^{-1},2)\cup(2,2+3^{-1})\implies V_{3^{-1}}(2)\setminus\{2\}\cap A = \varnothing\] Therefore fix any $x\in ([0,1]\cup[3,4])^c$, and suppose $x\neq 2$. By openness again, there exists an $\varepsilon>0$ such that
    \[V_\varepsilon(x)\subseteq \biggl([0,1]\cup[3,4]\biggr)^c\]
    Now as a band-aid fix, let $\delta = \min\{\varepsilon,d(x,2)\}2^{-1}$, so $V_\delta(x)\subseteq V_\varepsilon(x)$ and
    \[V_\delta(x)\subseteq \biggl([0,1]\cup\{2\}\cup[3,4]\biggr)^c\implies V_\delta(x)\cap\biggl([0,1]\cup\{2\}\cup[3,4]\biggr)=\varnothing\]
    Continuing crunching the $\varepsilon-\delta$s, we get
    \[\varnothing\supseteq V_\delta(x)\setminus\{x\}\cap A\supseteq \varnothing\]
    and $x\notin \acc{A}$. Therefore $\acc{A}=[0,1]\cup[3,4]$.
\end{proof}
\end{document}